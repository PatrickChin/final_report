
\section{Theory}

% \section{Single Particle Dynamics}
% \label{sec:single_particle_dynamics}

% To derive the method used in simulating the beam, it is usefull to start with
% Maxwell's equations. These can be used to describe the motion for charged
% particles under the magnetic fields of accelerator components, such as dipoles
% and quadrupoles. The solutions to these equations become increasingly complex at
% higher perturbations so it is usefull to devise a coordinate system other than

% Specifying a coorindate system such that the origin follows the ideal path of
% the particle beam, rather than a coordinate system about an arbitary fixed
% point we are able to describe the state of a patricle by

% \begin{equation}
% 	\begin{pmatrix}
% 		x(z) \\ x'(z) \\
% 		y(z) \\ y'(z) \\
% 	\end{pmatrix}
% 	=
% 	\begin{pmatrix}
% 		C_x(z)  & S_x(z)  & 0 & 0 \\
% 		C'_x(z) & S'_x(z) & 0 & 0 \\
% 		0 & 0 & C_y(z)  & S_y(z)  \\
% 		0 & 0 & C'_y(z) & S'_y(z) \\
% 	\end{pmatrix}
% 	\begin{pmatrix}
% 		x_0 \\ x'_0 \\
% 		y_0 \\ y'_0 \\
% 	\end{pmatrix}
% \end{equation}

% where \(x\) and \(y\) are deviations form the centre of the beam along their
% respective axis and \(x'\) and \(y'\) are the transverse momenta of the
% particle perpendicular to \(z\), the direction of travel of the beam.
% Transformations in either the $x$ or $y$ plane are independent, however, it is
% clear that coupling effects can still be included
% % TODO

\subsection{Single Particle Dynamics}

When working with beams of particles, it is advantagous to work in the
coordinate system that follows the ideal path of the beam. If the beam's motion
in the \(x\) and \(y\) planes are independent, i.e. we ignore coupling terms,
the each particle's motion in each plane can be described by
\begin{equation}
	\begin{pmatrix}
		u(z) \\ u'(z)
	\end{pmatrix}
	=
	\begin{pmatrix}
		C_u(z)  & S_u(z)  \\
		\sqrt{C'_u(z)} & S'_u(z)
	\end{pmatrix}
	\begin{pmatrix}
		u_0 \\ u'_0
	\end{pmatrix}
\end{equation}
where \(u\) is either \(x\) or \(y\) and \(u'\) is the transverse velocity of
the particle in the \(u\) plane. The coordinate \((u, u')\) lies in what is
known as phase space. Using this system of matricies the drift and quadrupole
matricies can be derived \cite{wiedemann2007particle}. The drift matrix is
\begin{equation}
	\mathcal{M}_D(l) =
	\begin{pmatrix}
		1 & l \\
		0 & 1
	\end{pmatrix} \\
\end{equation}
the focusing quadrupole matrix is
\begin{equation}
	\mathcal{M}_{QF}(l) =
	\begin{pmatrix}
		\cos\psi & \frac{1}{\sqrt{k}}\sin\psi \\
		-\sqrt{k}\sin\psi & \cos\psi
	\end{pmatrix} \\
\end{equation}
and the defocusing quadrupole matrix is
\begin{equation}
	\mathcal{M}_{QD}(l) =
	\begin{pmatrix}
		\cosh\psi & \frac{1}{\sqrt{\abs{k}}}\sinh\psi \\
		-\sqrt{\abs{k}}\sinh\psi & \cosh\psi
	\end{pmatrix}
\end{equation}

These transport matricies can be multiplied together resulting in the
transformation matrix representing a path containing all accelerator components
making it simple to follow a particle through a transport line.

\subsection{Emittance}

Grouping the individual particles in a particle beam, they will occupy an area
in phase space known as the emittance.

The beam emittance is a
% quantity that describes the collective motion of all the particles in the
% beam, providing a
qualitative way of describing the quality of the beam, essentially, it is a
measure of how parallel the particles of the beam are to each other.
% It is a conserved quantity in the absence of a \(z\) component (i.e. in the
% direction of the beam) in the magnetic field and when the beam is not being
% accelerated.

% The position of each particle in Cartesian coordinates is not sufficient in
% describing

% The state of a beam can be described by it's position and velocity, so each
% particle in the beam is represented in six-dimensional phase space with
% coordinates \(\left(x,p_x,y,p_y,z,p_z\right)\) where \(p\) is the momentum in
% it's respective direction. For very small 
% % TODO aaaaaaaaaaaaaaaaaaaaaaaaaaaaaaaaa do i need to derive this shit?
% where \(p_x\approx~p_0x'\) and
% \(p_y\approx~p_0y'\) are the transverse momenta, \(z\) is the position along the
% beam trajectory, \(p_z\) is the longitudinal momentum and \(x'\) and \(y'\) are
% the trajectory angles to the horizontal and vertical planes. Since the
% transverse momenta, and therefore \(x'\) and \(y'\), are generally quite small
% we can approximate \(\sin\left(x'\right)\approx x'\) and
% \(\sin\left(y'\right)\approx y'\). We can then project this six-dimensional
% volume into three independent two-dimensional phase planes, because in this
% approximation there is no coupling between those degrees of freedom.

The horizontal emittance of the beam is defined by considering the ellipse in
the \(x'-x\) phase space that contains \(95\%\) of all the
particles~\cite{buon1994beam}. The area contained by this ellipse divided by
\(\pi\) is defined as the emittance in units of \(\pi\)-mm-mrad.

\begin{equation}
	\int_{\text{ellipse}}\mathrm{d}x\mathrm{d}x' =\pi\epsilon
\end{equation}

\begin{figure}[!t]
	\centering
	\includegraphics{figures/ellipse}
	\caption{
		Graphical representation of the relation between the twiss parameters
		\cite{caldwell2015rkk}}
	\label{fig:ellipse}
\end{figure}

Fig. \ref{fig:ellipse} shows a beam projected onto a two dimensional phase
plane. The emittance can be described by the equation of the ellipse:

\begin{equation}
	\gamma x^2 + 2\alpha xx' + \beta x'^2 = \epsilon
\end{equation}

where \(\alpha\),  \(\beta\) are \(\gamma\) are ellipse parameters that
determine the ellipse's shape and orientation and are related by this equation

\begin{equation}
	\beta\gamma - \alpha^2 = 1
\end{equation}

% \subsection{Methods of measurement}

% The phase-space density and emittance of a beam must be infered from beam
% profiles captured using charge-coupled device (CCD) cameras after undergoing
% spatial filtering.

It follows that the beam matrix is

\begin{equation}
	\sigma =
	\begin{pmatrix}
		\sigma_{1,1} & \sigma_{1,2} \\
		\sigma_{2,1} & \sigma_{2,2} \\
	\end{pmatrix}
	=
	\begin{pmatrix}
		\beta\epsilon^2 & -\alpha\epsilon^2 \\
		-\alpha\epsilon^2 & \gamma\epsilon^2 \\
	\end{pmatrix}
\end{equation}

\noindent such that \(\epsilon = \det\sigma\), \(\sigma_{1,1}\) is the besm size
and \(\sigma_{1,2}\) is the orientation in phase space.

% We can relate the beam
% matrix after passing through quadrupoles, \(\sigma_1\), to the original beam
% matrix, \(\sigma_0\), as follows

% \begin{equation}
% 		\sigma_{1,11} = c^2(k)\sigma_{0,11} + 2c(k)s(k)\sigma_{0,12} +
% 		s^2(k)\sigma_{0,22}
% \end{equation}

% where we plot the final beam size \(\sigma_1,11\) against the quadrupole
% strength \(k\).  \(\sigma_0\) is determined from the fit and it's determinant
% will then give the emittance.

