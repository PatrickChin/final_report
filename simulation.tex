
\section{The Simulation}

The simulation of this experiment was split into three logical parts: the
simulation of the beam, the background and camera readout simulation, and the
reconstruction of the beam.
% the following is bs
It is designed such that each part of the code is able to act indepentently.

\subsection{The Electron Beam}

Given enough computing power and time, the simulation of the beam exiting the
plasma cell, passing through two quadrupoles and through a dipole could have
been done on BDSIM, a Geant4 toolkit for simulating radiation traveling through
an accelerator. This software package takes a particle and updates it's
position and velocity each step through the accelerator, applying the affect of
forces from all fields within the accelerator.  For beams of \num{1e9}
electrons, tracking each particle as it travels down the beamline is
impractical, therefore a new software was written that simulated the beam as
a whole.

\subsubsection{BDSIM calibration}

% TODO why use bdsim
% For a large sampe size
% got get
% Many particles will take very similar paths along the beam line.

The effect of the quadrupole and the dipole are dependant on the energy of the
individual electrons in the beam.

The dipole spreads the beam accross the screen splitting it into multiple beams
each with equal energies. The spreading of this beam is discretised, depending on
the This spreading is neither linear or quadratic and in
fact is 


BDSIM was used to simulate the firing of \num{100 000} single electrons down
the AWAKE beamline. These electrons had a square energy distribution, from
\SIrange{0}{10}{\tera\electronvolt} and no transverse momentum meaning no
emittance. These were used to plot 

Since the effect of the dipole on the beam depends on the energy of each
individual particle, the evolution of the beam through the dipole cannot be
calculated reasonably using beam matrices. So, to simulate the dipole, 

\subsubsection{Deriving the Beam Size Function}

The root mean square of the vertical beam size on the screen can be extracted
from the resultant beam matrix \ref{}.

The beam matrix element \(\sigma_{11} = \langle y^2 \rangle = \epsilon\beta\)

Multiplying the transport matricies for the quadrupoles and drifts

% \begin{align*}
% 	\bm{\sigma}_1 &=
% 	\epsilon
% 	\begin{pmatrix}
% 		\beta & -\alpha \\
% 		-\alpha & \gamma
% 	\end{pmatrix} =
% 	\mathcal{M}\;\bm{\sigma}_0\;\mathcal{M}^T, \quad
% 	% \text{where}\quad
% 	\epsilon = \sqrt{\det{\bm{\sigma}}}, \quad
% 	\beta\gamma-\alpha^2=1, \quad
% 	\tan2\varphi = \tfrac{2\alpha}{\gamma-\beta}
% 	\\[1em] %
% 	\mathcal{M} &= \underbrace{
% 	\begin{pmatrix}
% 		1 & d \\
% 		0 & 1
% 	\end{pmatrix}}_\text{drift}
% 	\cdot
% 	\underbrace{
% 	\begin{pmatrix}
% 		\cos\varphi_1 & \tfrac{\sin\varphi_1}{\sqrt{k_1}} \\
% 		-\sqrt{k_1}\sin\varphi_1 & \cos\varphi_1
% 	\end{pmatrix}}_\text{vertically focusing quadrupole}
% 	\cdot
% 	\underbrace{
% 	\begin{pmatrix}
% 		1 & g_2 \\
% 		0 & 1
% 	\end{pmatrix}}_\text{gap}
% 	\cdot
% 	\underbrace{
% 	\begin{pmatrix}
% 		\cosh\varphi_2 & \tfrac{\sinh\varphi_2}{\sqrt{|k_2|}} \\
% 		-\sqrt{k_2}\sinh\varphi_2 & \cosh\varphi_2
% 	\end{pmatrix}}_\text{horizontally focusing quadrupole}
% 	\cdot
% 	\underbrace{
% 	\begin{pmatrix}
% 		1 & g_1 \\
% 		0 & 1
% 	\end{pmatrix}}_\text{gap}
% \end{align*}


% - Counting all of the backgroudn electrons as error did not increase the
%   error by that much. I think the 
% - why can't dipole be accounted for in the transport matricies
% - what are the reasons for the quadrupoles?
% - was i correct in thinking that we assumed the quadrupoles were set to focus
% the mean beam energy? how will this work in the real spectrometer?
% - in the out.root file (the calibration file) the primaries tree has 100 000 
% entries whereas the prescreen sampler has 57 376. why is this?
% - this may be a stupid question but just to clear things up, when creating
%   the first histogram for each strip of x pixels (i.e. the histogram showing
%   the distribution of the electrons on the screen) why is there an error
%   along with this value? isn't this the exact number of electrons that hit
%   the screen in this simulation and all the error should come from the
%   background etc.
% - is there anything important about your code that I missed in my presentation?



% \begin{itemize}
% 	\item code developed to simulate the electron beam travellilgn from the
% 		iris of the plasma to t
% 	\item bdsim too slow to simulate 1e9 electrons individually
% 	\item instead 100 000 electrons were simulated in bdsim to create a
% 		reference function between energy and final horizontal position
% \end{itemize}

After generating, a two dimensional histogram representing the number of
electrons hitting the screen at each pixel the goal is to simulate the
effectiveness of the equipment and translate this number to represent the raw
signal that will be read off each pixel.

\subsection{Backgrounds}

How well the measurement of the emittance is, is most dependant on the
magnitude of the multiple sources of backgrounds as well as the reliability of
the equipment. The following sources of error were taken into account: the
efficiency of the scintillator screen, the acceptance of the camera, the
background photon density, the emittance of photoelectrons, the thermal noise,
the microchannel plate (MCP) and the readout noise. Each source of noise is added
to each pixel independently.

% TODO ask about the acceptance value.
% do we assume scintillator emits photons evenly?
% how far is the camera from the screen - I should know this
% what is the acceptance of the camera?
% where does the value 1e5*64/80 come for the background photon density?
% where do we assume the background photon density comes from

The first two error sources, the scintillator screen and the camera acceptance,
both scale the signal. So for each electron that hits the screen, it is
expected that an average of \num{5000} \cite{} photons are to be emitted. The
camera acceptance, is the ratio of photons that the camera registers to the
number of photons emitted by the scintillator, with a value of \num{1.5e-5}.
After the addition of these two effects, the camera is expected to receive
\SI{7.5}{\percent} of the original electron signal. The expected value for the
number of photons incident on the camera due to the beam electrons is a Poisson
random number.

It is assumed that there is a uniform distribution of photons incident on the
camera. The density of these electrons is expected to be
\SI{1e5}{\per\meter\squared} equating to \num{0.01} background photons per
pixel during the \SI{3e-3}{\second} the gate is open. This is a discrete
value, and so is also a Poisson random number. As discussed later in
Section~\ref{sec:results} this value is very small in comparison to the signal
produced by the beam and will only have an effect if the density of background
photons is multiple magnitudes larger than the expected value.

The camera's photomultipliers then convert the photons of light back to an
electrical current. This multiplies the incident number of photons by the
quantum efficiency of the camera, \num{0.15} \cite{}. The expected number of
thermal photoelectrons per pixel per second is expected to be \num{0.016} at
\SI{-30}{\celsius} with \SI{16}{\celsius} cooling water and an ambient room
temperature of \SI{16}{\celsius}. This value is typically doubles for each
\SI{5}{\celsius} rise in temperature of the camera.

% TODO lookup Intensified CCD Cameras
The microchannel plate amplifies the number of photoelectrons by \num{1442}
amplifying all previously added backgrounds.
This was simulated by simply scaling the value of the bin by this value rather
than generating a Poisson random number centred about this value.
% TODO why is a Poisson number not generated for MCP?

And finally, before the values of the signal is obtained, a readout noise is
added. This backgorund is expected to add \num{7.2} readout electrons per image
pixel for the camera operating at \SI{1}{\mega\hertz}.
% TODO see camera description for the readout noise

\subsection{Error Calculations}

Poisson statistics were used for the calculation of errors.  Once the shape of
the incident beam on the screen was calculated the number of electrons incident
on each pixel was given an error of the square root of the count. Two methods
of error propagation were used depending on the nature of the process involved.
The following processes are additive: the background photons, the thermal
photoelectrons and the readout noise, whereas the multiplicative processes are:
photon generation at the scintilator screen, photoelectron generation in the
camera PMTs and the amplification of the electron signal by the MCP.

Basic error propagation techniques were used here. For the additive processes,
where the new value of each bin $n$ is the sum between the old bin value $n_0$
and the value given by the process $n_\text{proc}$: $n=n_0+n_\text{proc}$ the
propagation of error is given by calculating the hypotenuse of the absolute
errors:

\begin{equation}
	\delta n= \sqrt{\delta n_0^2 + \delta n_\text{proc}^2}
\end{equation}
%
where the error of a Poisson random number is the square root of the value.

For the multiplicative processes, i.e. $n = \lambda_\text{proc}n_0 $ where
$\lambda_\text{proc}$ is the scaling factor of the process the propagation of
the error is given by calculating the hypotenuse of the percentage errors:

\begin{equation}
	\delta n= n \sqrt{
		\left( \frac{\delta n_0}{n_0} \right)^2 +
		\left( \frac{\delta n_\text{proc}}{n_\text{proc}} \right)^2}
\end{equation}

% TODO what is dn_proc ?? Why is the error dependant on the value of the
% background which is an unknown when this is done practically. Shouldn't 
% the background be the sqrt of the expected value?

% TODO simply scaling the errors up with the MCP correct?

\subsection{Binning errors}


%TODO talk about the systematic error due to discrete binning
