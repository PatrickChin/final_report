
\section{Results}
\label{sec:results}
% {{{

\begin{figure*}[!tb]
	\begin{minipage}[t]{\columnwidth}
			\centering
			\includegraphics[width=\columnwidth]{./output/run-bgdens/bgphotons_1/n1/graph.pdf}
			\caption{
				The beam reconstruction (blue line) of a sumlation run with all the
				expected parameter values. \(E = \SI{1.3}{\giga\electronvolt}\),
				\(\sigma_E = \SI{0.4}{\giga\electronvolt}\), \(\epsilon =
				\SI{1}{\milli\meter\milli\radian}\)
			}
			\label{fig:default}
	\end{minipage}\hfill
	\begin{minipage}[t]{\columnwidth}
			\centering
			\includegraphics[width=\columnwidth]{./output/run-pespread/pespread_0.01/n1/graph.pdf}
			\caption{
				The beam reconstruction, consistently overestimates the vertical
				beam size. This run used a small percentage energy spread of
				\SI{1}{\percent}. With all other parameters set to their expected
				value.
			}
			\label{fig:yoverestimate}
	\end{minipage}
\end{figure*}

\begin{figure*}[!tb]
	\begin{minipage}[t]{\columnwidth}
		\centering
		\includegraphics{./output/run-pespread/emit_vs_pespread.pdf}
		\caption{
			Plot of the simulated emittance measurement against the percentage
			spread of the beam energy, showing emittance measurements becoming
			unreliable at percentage energy spreads below \SI{2}{\percent}.
		}
		\label{fig:emit_pespread}
	\end{minipage}\hfill
	\begin{minipage}[t]{\columnwidth}
		\centering
		\includegraphics{./output/run-pespread/emitperr_vs_pespread.pdf}
		\caption{
			Plot of the simulated emittance measurement errors against the
			percentage spread of the beam energy, showing an exponential increase in
			the spread of the errors as the percentage error spread is narrowed.
		}
		\label{fig:emitperr_pespread}
	\end{minipage}
\end{figure*}

\begin{figure*}[!tb]
	\begin{minipage}[t]{\columnwidth}
		\centering
		\includegraphics{./output/run-inemit/emitr_vs_inemit.pdf}
		\caption{
			Plot of the ratio between the measured and true emittances of the beam
			against the true emittance of the beam.
			% , showing good measurements of the emittance below an emittance of
			% \SI{e-5}{\meter\radian} and a large underestimation of the emittance
			% for larger emittances.
			The blue line is the expected measurement value when taking into account
			the systematic overestimation due to discrete bins.
		}
		\label{fig:emitr_inemit}
	\end{minipage}\hfill
	\begin{minipage}[t]{\columnwidth}
		\centering
		% \includegraphics{./output/run-inemit/emitrperr_vs_inemit.pdf}
		\includegraphics[width=\columnwidth]{./output/run-inemit/emit_7e-5/n1/graph.pdf}
		\caption{
			Beam reconstruction for a large beam emittance of
			\SI{7e-5}{\meter\radian} showing the underestimation of the measured
			vertical beam sizes.
		}
		\label{fig:large_emit}
	\end{minipage}
\end{figure*}

\begin{figure*}[!tb]
	\begin{minipage}[t]{\columnwidth}
		\centering
		\includegraphics{./output/run-bgdens/emit_vs_bgdens.pdf}
		\caption{
			Plot of the measured beam emittance against a factor of the expected
			background density of \SI{3.415e4}{photons\per\meter\squared}
		}
		\label{fig:emit_bgdens}
	\end{minipage}\hfill
	\begin{minipage}[t]{\columnwidth}
		\centering
		% \includegraphics{./output/run-bgdens/emitperr_vs_bgdens.pdf}
		\includegraphics[width=\columnwidth]{./output/run-bgdens/bgphotons_1e4/n6/graph.pdf}
		\caption{
			Beam reconstruction for a background \num{1e4} times the expected
			background photon density.
		}
		\label{fig:large_bg}
	\end{minipage}
\end{figure*}

Along with the \(\chi^2\) minimised parameter values of the fit, each simulation
generated a plot, showing the simulated, measured and fitted vertical beam size
functions as a function of horizontal position \(x\) on the screen.
Figure~\ref{fig:default} is the output of a run with all parameters set to their
expected values.  Figure~\ref{fig:yoverestimate} is an output plot for a run
with a small (\SI{1}{\percent}) energy spread so the spread of the beam is very
narrow across the screen, so this plot is essentially zoomed in at a small
section of the screen. The solid black line is the shape of the simulated
electron beam that hits the screen, the black points show the simulated
measurements of the RMS width of the fitted Gaussian for each vertical strip of
pixels. The blue dashed line is the beam size function fitted to the points.
% }}}

\subsection{Binning errors}
% {{{

After the investigation of multiple experimental parameters, the emittance
measurement consistently converged to a value \num{1e-8} larger than the input
emittance. The reason for this systematic error was found to be due to the
discretization of the beam hitting the screen meaning that the measurement of
the vertical beam size was consistently overestimated. Since the electrons in
each pixel are not uniformly distributed but rather more densely distributed
closer towards the mean value, giving rise to a systematic overestimation of the
vertical beam size of up to two times the vertical size of the pixel. This
effect can be seen most clearly when a very small energy spread was used as can
be seen in Figure~\ref{fig:yoverestimate}, where the measured beam heights are
consistently larger than the actual beam height.

This systematic error is displayed in subsequent plots as a blue line, where the
red line shows the true beam emittance and the blue line represents where the
emittance measurement should be taking into account this error.
% }}}

\subsection{Energy Spread}
% {{{

Initially, the mean energy of the beam and energy spread of the beam were tested
independently. Simulations for all combinations of the following energies \(E
\in \left\{ 0.5, 1, 1.3, 2.0, 3.0 5.0\right\} \) and the following energy
spreads \(\sigma_E \in \left\{ 0.01, 0.1, 0.3, 0.4 \right\}\) were run.  These
energies and energy spreads were chosen such that at least one full standard
deviation of the beam hit the screen. As Figure~\ref{fig:eofx} shows, the range
of energies that hit the screen for this setting of the dipole and quadrupole,
is from \SIrange{\sim0.28}{6}{\giga\electronvolt}.
% TODO what is the dipole set to?

The smaller the energy spread of the beam the smaller the spread of the beam
across the screen. Figure~\ref{fig:yoverestimate} is a plot showing the full
spread of the beam across the screen for a beam energy spread of
\SI{1}{\percent}. The fewer vertical beam measurements that the function is able
to fit to, the larger the errors of fitting will be, so the relationship between
the errors of the measured emittance and the energy spread in
Figure~\ref{fig:emit_pespread} is the expected result. The expected energy
spread, \SI{0.4}{\giga\electronvolt}, translates to a percentage energy spread
of \SI{30}{\percent} at the expected beam energy of
\SI{1.3}{\giga\electronvolt}. By this point, the error on the measurement of the
emittance has converged to less than a \SI{1}{\percent}.

Plotting the absolute simulated measurement error against the percentage energy
spread in Figure~\ref{fig:emit_pespread} it is clearer the manner in which the
errors blow up for lower energy spreads. 
% This was not the expected result. % Should I say this?
For lower energy spreads, all measurement errors were expected to increase
exponentially.  However this is not the case, but rather, the \emph{spread} of
measurement errors increased exponentially, meaning that may measurement errors
are still only a few percent of the measurement. This behaviour reflects how the
errors for the background noise were calculated; the error associated with the
background photons for each pixel is set to the square root of the number of
background photons. This means the fewer the number of incident background
photons the screen, the smaller the error. However, the error in the uncertainty
of the measurement of the background was not taken into account then scaling the
raw signal back into the real shape. This extra error arises from the
uncertainty in the measurement of the background photon density when there is no
accelerated election beam. To take this into account, this error must be added
in quadrature to each pixel.

The upper ranges of this energy spread may also be investigated, however, this
is less of a priority. Percentage energy spreads up to \SI{80}{\percent} were
investigated without signs of alterations to the measured emittance of emittance
measurement. It is expected that as the beam energy moves outside the
\SIrange{0.28}{6}{\giga\electronvolt} range, emittance measurements will become
more erroneous since most of the beam will not hit the screen.
% }}}

\subsection{Input Emittance}
% {{{



Since the emittance is the parameter required to be measured, a reasonably large
region around the expected emittance should give precise emittance measurements.
The input beam emittance range tested was from \SIrange{1e-7}{1e-4}%
{\meter\radian} as shown in Figure~\ref{fig:emitr_inemit}. Emittances below
\SI{e-5}{\meter\radian} showed accurate emittance measurements, with the
\SI{\sim e-8}{\meter\radian} overestimation of the emittance measuring
persisting over this range.

Increasing the emittance of the beam above \SI{e-5}{\meter\radian} results in an
underestimation in the measurement of the emittance as well as an increase in
the measurement's error. This behaviour is expected. Increasing the emittance of
the beam means the spread of the beam on the screen is larger. At emittances
above \SI{e-5}{\meter\radian}, a significant proportion of the beam's electrons
no longer hit the \SI{6.5}{\centi\meter} tall screen, so particles with the
largest transverse momentum no longer contribute to the emittance measurement.
% TODO is this sentence ok??
Figure~\ref{fig:large_emit} displays this effect showing that the shape of the
beam on the screen is a lot less focused in the vertical axis and each vertical
beam size measurement is underestimated in comparison to the run of expected
parameters in Figure~\ref{fig:default}.

% This simulation shows that the emittance can be determines accurately bellow

% }}}

\subsection{Background Photons}
% {{{

This background noise is the most likely source of error to change during and
between runs, as all other sources of error arise from intrinsic properties of
the equipment, or from the setup of the experiment. The background photon noise
is expected to be almost insignificant in comparison to the signal of the beam,
where the ratio of signal to background photons is expected to be in the order
of magnitude \num{4e4}. The aim was to find the level of background photon
radiation at which the measurement of the emittance would be strongly affected.
The expected background density was multiplied by an arbitrary factor until the
emitance value deviated from the true value.  Figure~\ref{fig:emit_bgdens} shows
how this factor affects the measurement. Only after the expected background has
beed multiplied by a factor of \num{\sim100} is any effect seen on the measured
emittance. Between background factors \num{e2} and \num{e3} the measured
emittance begins to be underestimated and above a background factor of \num{e3}
the emittance measurement becomes increasingly imprecise with a deviation a few
percent at a background factor of \num{2e3}, increasing up to \SI{10}{\percent}
at a background factor of \num{e7}.

The reasoning for a larger spread in the measured emittance values comes from
the background drowning out the signal, that is, the fluctuations of the
randomly generated background between the pixels become large enough to distort
the shape of the image on the screen.  The amount by which the image is distored
is not taken into account by the error bars in Figure~\ref{fig:emit_bgdens},
suggesting that error in the background was not completely accounted for. The
absence of this error is propagated to the error of the emittance measurement,
as can be seen in Figure~\ref{fig:emit_bgdens}, as the error significanly
underestimates the fluctuations in the measurement.  However, these error bars
can be inferred from the spread of the measurements since for lower backgrounds
all ten runs are closely grouped and for higher backgrounds there is a clear
spread of measurement values.

% % TODO explain why measured emittance is always underestimated
% % The reasoning behind the emittance measurement being underestimated
% % arises from the background affecting a shallower signal more than a sharp one.
% Figure~\ref{fig:emit_bgdens}, also shows that the emittance measurement is
% consistently underestimated for large backgrounds. This can be explained by the
% method of calculating the beam size for a single strip of pixels.
% % TODO pol0+gaus


% }}}

