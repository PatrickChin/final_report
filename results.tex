
\section{Results}
\label{sec:results}

Along with the fitted parameter values of the fit, each simulation output a
plot, showing the simulated, measured and fitted values for the vertical beam
size against the horizontal position on the screen. The solid black line is the
shape of the simulated electron beam expected to hit the screen, the black
points with error bars are the
% TODO this is a mess
measured root mean squared spread of the gaussian distributed vertical beam size
of the

\subsection{Binning errors}

\begin{figure}[tb]
	\centering
	\includegraphics[width=1\linewidth]{./output/run-pespread/pespread_0.01/n1/graph.pdf}
	\caption{
		Blue line representing the reconstruction of the shape of the beam that
		hits the screen consistently overestimates the vertical beam size. This
		run used a small percentage energy spread of \SI{1}{\percent}. With all
		other parameters set to their expected value.
	}
	\label{fig:yoverestimate}
\end{figure}

After the investigation of multiple experimental parameters, the emittance
measurement consistently converged to a value \num{1e-8} larger than the input
emittance. The reason for this systematic error was found to be due to the
discritisation of the beam hitting the screen meaning that the measurement of
the vertical beam size was consistently overestimated. Since the electrons in
each pixel are not uniformly distributed but rather more densely distributed
closer towards the mean value, giving rise to a systematic overestimation of the
vertical beam size of up to two times the vertical size of the pixel. This
effect can be seen most clearly when a very small energy spread was used as can
be seen in Figure~\ref{fig:yoverestimate}, where the measured beam heights are
consistently larger than the actual beam height.


\subsection{Energy Spread}

% Initially, the mean energy of the beam and energy spread of the beam were tested
% independently.
It was found that the error of the measurement of the emittance 
% TODO oh god this was the wierd one

\subsection{Input Emittance}



\subsection{Background Photons}
