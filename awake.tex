
\section{AWAKE}

The aim of this experiment is to provide a proof-of-concept for proton driven
plasma wakefield acceleration to be able to accelerate electrons beams to the
\si{\tera\electronvolt} energy scale. An overview of the experiment as described
in the AWAKE design overview: \emph{AWAKE, The Advanced Proton Driven Plasma
Wakefield Acceleration Experiment at CERN}~\cite{Gschwendtner2015rni} is as
follows:

The CERN Super Proton Synchrotron (SPS) will provide a
\SI{400}{\giga\electronvolt} proton beam with a bunch length of \(\sigma_z =
\SI{12}{\centi\meter}\) and an intensity of \(\sim \num{3e11}\) protons per
bunch. This will travel down the \SI{750}{\meter} long proton beam line,
previously used for the CERN Neutrinos to Gran Sasso project (CNGS), and be
focused to a horizontal and vertical rms size \(\sigma_{x,y} =
\SI{200}{\micro\meter}\) before entering a \SI{10}{\meter} long Rubiduim vapor
plasma cell with an adjustable density at the \num{e14} to \num{e15}
\si{electrons\per\centi\meter} scale.

The proton driver will self modulate at the plasma wavelength \(\lambda_{pe}\)
after being seeded by a high powered \(\approx \SI{4.5}{\tera\watt}\) laser
pulse that is co-axial and co-propagating with the proton driver beam. This
laser also serves the purpose of ionising the Rubidium vapor to create the
plasma. For these beams to be co-axial for the full length of the plasma cell,
they need to be synchronous to within \SI{100}{\pico\second} and the size of the
focal point of the proton beam is required to be \(\le\SI{100}{\micro\meter}\)
and \(\le\SI{15}{\micro\radian}\)

The electron witness beam will be created via photo-emission by an illuminating
cathode electron source and accelerated by a 2.5 cell RF-gun and a meter long
booster at \SI{3}{\giga\hertz}.

%TODO injection

% \section{Project outline}

% \lipsum[3-10]

% The development of this experiment has been heavily simulation driven.
% Simulation code developed specifically for the simulation of the plasma to be
% able to resolve for time scales of \(\omega_p^{-1}\), (where \(\omega\) is the
% frequency of the plasma wave) and length scales of down to \(c/\omega_p\), as
% existing codes were not tuned to resolve at these scales.  Different simulation
% softwares are tuned to be used for different sections of the AWAKE experiment.

% I will be working on simulating the electron spectrometer using
% BSDIM~\cite{agapov2009bdsim}, simulation software in active development,
% designed to simulate and track particle beams passing through accelerators and
% detectors. It is built on top of the Geant4
% toolkit~\cite{agostinelli2003geant4} for the simulation of particles through
% matter, which also provides the graphical user interface for a visualisation of
% the simulation.  Event data is stored using ROOT~\cite{antcheva2011root}, an
% advanced statistical analysis and visualisation framework designed to work for
% petabyte scale data storage.

% More specifically, I will initially be looking at calculating the emittance of
% the accelerated electron beam using data from simulated accelerated electron
% beams.  Recent simulations of the spectrometer used an idealised electron beam
% \cite{deacon2016qjq} and I will be continuing this line of investigation.  The
% electron beam profile and other properties immediately after it leaves the
% plasma cell will be provided by separate simulations using LCODE.  This data is
% used as input for the BDSIM simulation where we will simulate the beam passing
% through dual focusing quadrupoles in both the horizontal and vertical planes.
% The simulation will be able to provide all the raw data about the final state
% of the electron beam, however, in reality we will not be able to simply query
% the beam properties. The measurement of the energy spectrum will be carried out
% by using a magnetic dipole downstream of the dual quadrupoles, and observing
% the horizontal spread of the electron beam on a screen. This screen will also
% be sumulated with BDSIM taking into account the screen resolution and detection
% rates.

% I will also be working on the modeling and simulation of the background
% radiation from the plasma cell and other sources using real world data to help
% build an accurate model.  All of these simultions along with real data will
% help in finding optimal parameters for each component of the spectrometer,
% including the strength of the quadrupoles and the dipole, the lens parameters
% of the camera and the properties of the screen.

