
\chapter{Project outline}

\lipsum[3-56]

The development of this experiment has been heavily simulation driven.
Simulation code developed specifically for the simulation of the plasma to be
able to resolve for time scales of \(\omega_p^{-1}\), (where \(\omega\) is the
frequency of the plasma wave) and length scales of down to \(c/\omega_p\), as
existing codes were not tuned to resolve at these scales.  Different simulation
softwares are tuned to be used for different sections of the AWAKE experiment.

% I will be working on simulating the electron spectrometer using
% BSDIM~\cite{agapov2009bdsim}, simulation software in active development,
% designed to simulate and track particle beams passing through accelerators and
% detectors. It is built on top of the Geant4
% toolkit~\cite{agostinelli2003geant4} for the simulation of particles through
% matter, which also provides the graphical user interface for a visualisation of
% the simulation.  Event data is stored using ROOT~\cite{antcheva2011root}, an
% advanced statistical analysis and visualisation framework designed to work for
% petabyte scale data storage.

% More specifically, I will initially be looking at calculating the emittance of
% the accelerated electron beam using data from simulated accelerated electron
% beams.  Recent simulations of the spectrometer used an idealised electron beam
% \cite{deacon2016qjq} and I will be continuing this line of investigation.  The
% electron beam profile and other properties immediately after it leaves the
% plasma cell will be provided by separate simulations using LCODE.  This data is
% used as input for the BDSIM simulation where we will simulate the beam passing
% through dual focusing quadrupoles in both the horizontal and vertical planes.
% The simulation will be able to provide all the raw data about the final state
% of the electron beam, however, in reality we will not be able to simply query
% the beam properties. The measurement of the energy spectrum will be carried out
% by using a magnetic dipole downstream of the dual quadrupoles, and observing
% the horizontal spread of the electron beam on a screen. This screen will also
% be sumulated with BDSIM taking into account the screen resolution and detection
% rates.

% I will also be working on the modeling and simulation of the background
% radiation from the plasma cell and other sources using real world data to help
% build an accurate model.  All of these simultions along with real data will
% help in finding optimal parameters for each component of the spectrometer,
% including the strength of the quadrupoles and the dipole, the lens parameters
% of the camera and the properties of the screen.

\section{The Simulation}

\begin{itemize}
	\item code developed to simulate the electron beam travellilgn from the
		iris of the plasma to t
	\item bdsim too slow to simulate 1e9 electrons individually
	\item instead 100 000 electrons were simulated in bdsim to create a
		reference function between energy and final horizontal position
\end{itemize}

\section{Binning errors}

Initial result
