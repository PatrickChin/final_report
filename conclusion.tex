
\section{Conclusion}
\label{sec:conclusion}

% In these simulations it was also assumed that the accelerated electron beam
% had 
In these simulations, certain properties of the accelerated electron beam were
approximated in order to simplify the simulation. These approximations assumes
that the beam has a Gaussian energy spread and that it's phase space
distribution was well defined by the enclosing ellipse. Whereas the energy
spread of the beam will more likely be skewed and it's distribution in phase
space will have `tails' extending from the central ellipse. The difference these
approximations make to the accuracy of the simulations carried out in this
report, is assumed to be quite small.  A change in the energy distribution will
only alter the density of electrons on the screen and the number electrons in
the `tails' of the phase space are expected to be three orders of magnitude less
dense than the central phase volume.

The effect of three different experimental parameters were investigated, and
the ranges between which the emittance measurements were accurate and reasonably
precice were found. Despite the fact that the error measurements were
misscalculated, these can be infered from the spread of these measurements.
So the behaviour of the emittance measurement with respect to these parameters
have been presented.

There still remains many possible combinations of these parameters to be
investigated. For example at higher backgrounds, the range of percentage energy
spreads where an emittance measurement would be reliable is likely to shift. For
this reason, the expected values for all other parameters that were not being
investigated were used.
