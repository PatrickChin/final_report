
\section{Conclusion}
\label{sec:conclusion}

% In these simulations it was also assumed that the accelerated electron beam
% had 
In these simulations, certain properties of the accelerated electron beam were
approximated in order to simplify the simulation. These include assumptions that
the beam will have a Gaussian energy spread and that it's phase space
distribution was well defined by the enclosing ellipse. The energy spread of the
beam, however, will more likely be skewed and it's distribution in phase space
will have `tails' extending from the central ellipse. The difference these
approximations make to the accuracy of the simulations carried out in this
report, is assumed to be quite small.  A change in the energy distribution will
only alter the density of electrons on the screen and the number electrons in
the `tails' of the phase space are expected to be about three orders of
magnitude less dense than the central phase volume~\cite{deacon2016qjq}.

The effect of three different experimental parameters were investigated, and the
ranges between which the emittance measurements were accurate and reasonably
precise were found. Despite the fact that the error measurements were
miscalculated, these can be inferred from the spread of these measurements, so
the behaviour of the emittance measurement with respect to these parameters have
been presented.

There still remains many possible combinations of these parameters to be
investigated. For example at higher backgrounds, the range of percentage energy
spreads where an emittance measurement would be reliable is likely to shift. For
this reason, all parameters not being investigated were set to their expected
values.
