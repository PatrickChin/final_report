
\chapter{Introduction}

Advancements in quantum and particle physics are driven by experimental
observations. They are used for verifying hypothesise based on previous
observations or for providing data from which new hypothesise can be drawn.
Particle colliders create large amounts of observation data from particle
interactions and improvements to these come largely in the form of increasing
the energy of the colliding particle beams. Proton beam energies have reached
energies of \SI{13}{\tera\electronvolt} using current synchrotrons, however
lepton-lepton colliders have yet to reach the \si{\tera\electronvolt} energy
scale.  Higher energy electron and positron colliders promise much higher
precision measurements as they are fundamental, point-like particles and they
will be able to interact within a much cleaner environment.

Looking at current accelerators, circular electron or positron accelerators are
not possible at these energies unless the accelerator reaches the
\SI{100}{\kilo\meter} scale as electrons at this energy approach the speed of
light and since they are accelerating in a circle they will radiate large
amounts of their energy. An accelerator at the \SI{100}{\kilo\meter} scale is
impractical due to geographical and financial limitations.  Similar problems
also arise in linear colliders where current radiofrequency (RF) cavities in a
linear collider will have to be tens of kilometers in length to reach the
\si{\tera\electronvolt} scale, this is with current acceleration gradients of
up to \SI{100}{\mega\volt\per\meter}.  This urges the development of a new
methods for the acceleration of particles.

\section{Proton Driven Plasma Wakefield Acceleration}

The concept of accelerating particles in plasma was promising as plasma is able
to sustain large electric fields.  The idea being that energy can be
transferred to a group of charged particles by injecting them into the plasma
wakefield that follows a high energy laser pulse or proton bunch, using the
plasma as an energy transfer medium.  The witness bunch is then accelerated by
the high electromagnetic gradient.

\subsection{Self-modulation instability}

The first challenge in the development of this accelerator was getting the
length of the proton driver bunch small enough so that resonance occurs with
the electrons in the plasma.  Typical proton bunches, i.e. those produced by
the CERN Super Proton Synchrotron (SPS), have lengths of \(\sim
\SI{10}{\centi\meter}\) which cannot directly create strong plasma waves at the
required wavelength in the \si{\milli\meter} scale as the Fourier component of
the proton beam at the plasma frequency is negligible.
Simulations~\cite{kumar2010self} on the compression of these bunches show that
reducing the longitudinal phase volume blows up the transverse phase volume.
An alternative method would be to split up the proton bunch into a number of
micro-bunches to be simultaneously decelerated.

An instability between the beam and the plasma arises from the mutual
amplification of the rippling of the beam radius and the plasma wave. This
instability tends to destroy the plasma wave as the amplification focuses and
defocuses selected slices of the beam.  This problem was solved by seeding the
self-modulated instability (SMI) with a short electron bunch
\cite{lotov2013natural}, a laser pulse \cite{siemon2013laser} or a sharp cut in
the bunch profile\cite{kumar2010self}. This will promote a single mode and
suppress other modes, including the strongest competing modes, the hosing modes
\cite{vieira2014hosing} and produce well-separated micro-bunches.

\subsection{Uniform-density plasma cell}

The plasma wavelength is \(\lambda_{pe} \approx \SI{1.26}{\milli\meter}\)
meaning that the \SI{10}{\centi\meter} proton bunch will have to be split into
\(\sim 100\) micro-bunches in order to be able to drive the wake.  Each
micro-bunch contributes to the wakefield, and only if the plasma density is
uniform will the contribution of each bunch be coherent. Incoherence will cause
the electron bunches to arrive at the wrong phase in the plasma oscillation. An
increase in the plasma density will shorten the plasma wavelength causing the
electron bunch to crest plasma wave it was riding and fall into the defocusing
phase of the plasma wave as shown in Fig \ref{fig:plasma_phase}(a). A decrease
in the plasma density will increase the plasma wavelength causing the plasma
wave to fall further behind the electron bunch meaning the electron bunch to
fall into the trough of the plasma wave resulting in a deceleration of the
electron beam \ref{fig:plasma_phase}(c). The electron beam must be in the
region of length \(\lambda_{pe}/4\) between the defocusing and decelerating
phases of the plasma wave.
% These effects are significantly larger for the electrons as protons have
% large longitudinal momentum.

\begin{figure}[!t]
	\centering
	% \includegraphics[width=0.7\linewidth]{res/plasma_phases.pdf}
	\caption{Phasing of the electron bunch for increased density (a) correct
		density (b) and decreased density (c). Figure credit to
		\cite{caldwell2015rkk}}
	\label{fig:plasma_phase}
\end{figure}

This requirement of the plasma limits the plasma selection to being uniform
rubidium vapor, ionised by a co-propagating laser
pulse~\cite{oz2014novel,oz2014bja}.  Rubidium was chosen due to it's low
ionization potential and heavy atomic mass.  A heavy element is required to
minimize the movement of the plasma's nuclei which causes adverse effects on
the plasma's behaviour \cite{vieira2012nj, vieira2014bqa}. The Rubiduim vapor
is kept in thermodynamic equilibrium at a constant temperature and volume.

\subsection{Injection of the witness beam}

Due to SMI, the shape of the drive beam changes in the plasma and for the first
four meters, the difference between the phase velocity of the wakefield and the
proton beam velocity is quite large and this will effect the electron beam in
the same manner as having a non uniform plasma, detailed above. To avoid this
problem it was suggested that the electrons could be injected into the plasma
after SMI had fully developed. The design of the injection method arrived at
passing the electron beam through a narrow vacuum tube separated from the
plasma by a thin foil. Then after \(\sim \SI{4}{\meter}\) the electrons will be
directed into the wakefield close close behind the proton driving beam.

\subsection{AWAKE}

The aim of this experiment is to provide a proof of concept for proton driven
plasma wakefield acceleration.  An overview of the experiment is as follows:

The SPS will provide the \SI{400}{\giga\electronvolt} proton driver beam with a
bunch length of \(\sigma_z = \SI{12}{\centi\meter}\) and an intensity of \(\sim
3\times 10^{11}\) protons/bunch. This will travel down the \SI{750}{\meter}
long CNGS transfer line and be focused to \(\sigma_{x,y} =
\SI{200}{\micro\meter}\) and enter a \SI{10}{\meter} long Rubiduim vapor plasma
cell with an adjustable density at the \(10^{14}\) to \(10^{15}\)
electrons/\si{\per\centi\meter} scale.

The proton driver will self modulate at the plasma wavelength \(\lambda_{pe}\)
after being seeded by a high powered \(\approx \SI{4.5}{\tera\watt}\) laser
pulse that is co-axial and co-propagating with the proton driver beam. This
laser also serves the purpose of ionising the Rubidium vapor. These two beams
need to be synchronous to within \SI{100}{\pico\second} and the focal point of
the proton beam is required to be \(\le\SI{100}{\micro\meter}\) and
\(\le\SI{15}{\micro\radian}\) so they are co-axial for the full length of the
plasma cell.

The electron witness beam will be created via photo-emission by an illuminating
cathode electron source and accelerated by a 2.5 cell RF-gun and a meter long
booster at \SI{3}{\giga\hertz}.

%TODO injection

\section{Emittance}

The beam emittance is a quantity that describes the collective motion of all
the particles in the beam, providing a qualitative way of describing the
quality of the beam. It is a conserved quantity in the absence of a \(z\)
component (i.e. in the direction of the beam) in the magnetic field and when
the beam is not being accelerated.

The position of each particle in Cartesian coordinates is not sufficient in
describing the state of a beam so each beam particle is represented in
six-dimensional phase space with coordinates \(\left(x,p_x,y,p_y,z,p_z\right)\)
where \(p_x\approx~p_0x'\) and \(p_y\approx~p_0y'\) are the transverse momenta,
\(z\) is the position along the beam trajectory, \(p_z\) is the longitudinal
momentum and \(x'\) and \(y'\) are the trajectory angles to the horizontal and
vertical planes. Since the transverse momenta, and therefore \(x'\) and \(y'\),
are generally quite small we can approximate \(\sin\left(x'\right)\approx x'\)
and \(\sin\left(y'\right)\approx y'\). We can then project this six-dimensional
volume into three independent two-dimensional phase planes, because in this
approximation there is no coupling between those degrees of freedom.

The horizontal emittance of the beam is defined by considering the ellipse in
the \(x'-x\) phase space that contains \(95\%\) of all the
particles~\cite{buon1994beam}. The area contained by this ellipse divided by
\(\pi\) is defined as the emittance in units of \(\pi\)-mm-mrad.

\[ \int_{ellipse}\mathrm{d}x\mathrm{d}x' =\pi\varepsilon \]

\begin{figure}[!t]
	\centering
	% \includegraphics[width=0.7\linewidth]{res/phase_space_old.pdf}
	\caption{Phase space ellipse with \(x\) or \(y\) on the horizontal axis
		and \(x'\) or \(y'\) on the vertical axis. (credit to
		\cite{wiedemann2007particle})}
	\label{fig:phase_space}
\end{figure}

Fig. \ref{fig:phase_space} shows a beam projected onto a two dimensional phase
plane. The emittance can be described by the equation of the ellipse:

\[ \gamma x^2 + 2\alpha xx' + \beta x'^2 = \varepsilon \]

where \(\alpha\),  \(\beta\) are \(\gamma\) are ellipse parameters that
determine the ellipse's shape and orientation and are related by this equation

\[ \beta\gamma - \alpha^2 = 1 \]

% \subsection{Methods of measurement}

% The phase-space density and emittance of a beam must be infered from beam
% profiles captured using charge-coupled device (CCD) cameras after undergoing
% spatial filtering.

It follows that the beam matrix is

\[\sigma =
	\begin{pmatrix}
	  \sigma_{1,1} & \sigma_{1,2} \\
	  \sigma_{2,1} & \sigma_{2,2} \\
	\end{pmatrix}
	=
	\begin{pmatrix}
	  \beta\varepsilon^2 & -\alpha\varepsilon^2 \\
	  -\alpha\varepsilon^2 & \gamma\varepsilon^2 \\
	\end{pmatrix}
\]

such that \(\varepsilon = \det\sigma\), \(\sigma_{1,1}\) is the besm size and
\(\sigma_{1,2}\) is the orientation in phase space. We can relate the beam
matrix after passing through quadrupoles, \(\sigma_1\), to the original beam
matrix, \(\sigma_0\), as follows

\[
	\sigma_{1,11} = c^2(k)\sigma_{0,11} + 2c(k)s(k)\sigma_{0,12} +
	s^2(k)\sigma_{0,22}
\]

where we plot the final beam size \(\sigma_1,11\) against the quadrupole
strength \(k\).  \(\sigma_0\) is determined from the fit and it's determinant
will then give the emittance.


