
\section{Introduction}

Advancements in quantum and particle physics are primarily driven by
experimental observations which can verify or refute previous hypotheses, or can
provide data from which new hypotheses can be drawn, with the overall goal
of helping us have a deeper understanding of the universe around us.
Particle colliders are a main source of observational data at the quantum scale,
and can create millions of collision events every second.  Design modifications
to these colliders mostly increase the luminosity of the collision in order to
increase the collision rate and produce more data.  This report however, focuses
on a design modification aimed at increasing the energy of the colliding beams.
Increasing the energy of particle colliders will give the ability to investigate
energy regions yet to be reached and allow the observation of interactions that
only happen at higher energies. These interactions may give insight into
questions pertaining to the unification of the fundamental forces.

Proton--proton beam energies at the Large Hadron Collider (LHC) have recently
reached energies of \SI{13}{\tera\electronvolt} \cite{CMS:2015bta}, whereas
lepton--lepton colliders have yet to reach the \si{\tera\electronvolt} energy
scale. The largest lepton--lepton collider, the Large Electron-Proton Collider
(LEP), was closed down to make way for the LHC in \num{2000} after having
reached a maximum energy of \SI{209}{\giga\electronvolt} \cite{Barate2003sz}.

The appeal of colliding leptons over composite particles such as protons arises
from the fact that leptons are fundamental point-like particles. Their
centre-of-mass energy is more easily determined and produce a much cleaner
environment on collision, allowing for easier analysis of data as less
interactions need to be taken into account.
% TODO improve this explanation

One of the drawbacks to circular accelerators,
% for lighter particles especially,
is the loss of a particle's energy due to synchrotron radiation.  This is the
emittance of radiation from relativistic charged particles that are moving in a
uniform magnetic field. The energy loss is inversely proportional to the fourth
power of the rest mass of the particle \cite{sokolov1966synchrotron}, meaning
that electrons lose more energy than protons by a factor of about \num{e13}
which is.  During experiments performed at the LEP the radiated power when
running at \SI{100}{\giga\electronvolt} reached about \SI{18}{\mega\watt} which
needs to be resupplied to the beam.
% on-top of the power needed to accelerate the beam.
% TODO ^
% TODO I'm not sure how much 18MW is.
% TODO should I talk about synchrotron radiation in the theory section?

% TODO talk more about the goals and location?
This continuous loss of energy can be overcome by creating linear particle
accelerators. There are currently two radio-frequency (RF) linear lepton--lepton
accelerator proposals, the Compact Linear Collider (CLIC)~\cite{Linssen2012hp}
and the International Linear Collider (ILC)~\cite{Behnke2013xla} which are
expected to reach collision energies of up to several \si{\tera\electronvolt}
and \SI{500}{\giga\electronvolt} respectively. Both collaborations have recently
joined efforts under the Linear Collider Collaboration. The continuous scaling
of linear accelerators to higher and higher energies, with current RF technology
requires greater accelerator lengths reaching lengths extending to the
\SI{100}{\kilo\meter} scale. Building accelerators of this scale is deemed
impractical for most situations due to geographical and financial limitations.
This urges the development of new technologies in order to continue pushing the
energy frontier of particle accelerators while scaling down the accelerator
lengths.


