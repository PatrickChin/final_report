
\section{Introduction}

Advancements in quantum and particle physics are primarily driven by
experimental observations, which can verify or refute previous hypotheses, or
can provide data from which new hypotheses can be drawn.  The overall goal is to
gain a deeper understanding of the universe around us.  Particle colliders are a
main source of observational data at the quantum scale, and can create millions
of collision events every second.  Design modifications to these colliders
mostly increase the luminosity of the colliding beams in order to produce more
data.  This report however, focuses on a design modification aimed at increasing
the energy of the colliding beams. This will allow the investigation of energy
regions that have yet to be reached, thus allowing the observation of particle
interactions that only happen at higher energies. These interactions may give
insight into questions pertaining to the unification of the fundamental forces.
% or possibly give us more questions to be answered.

Beams at the Large Hadron Collider (LHC) operated at centre-of-mass energies of
\SI{7}{\tera\electronvolt} and \SI{8}{\tera\electronvolt}~\cite{Flechl2015xxa}
in the quest to find the Higgs Boson, and proton--proton centre-of-mass energies
at CERN have recently reached \SI{13}{\tera\electronvolt}~\cite{CMS:2015bta}.
Lepton--lepton colliders, however, have yet to reach the \si{\tera\electronvolt}
energy milestone. The largest lepton--lepton collider, the Large
Electron-Proton Collider (LEP), was closed down to make way for the LHC in
\num{2000} after having reached a maximum energy of
\SI{209}{\giga\electronvolt}~\cite{Barate2003sz}.

The appeal of colliding leptons over composite particles such as protons arises
from the fact that leptons are fundamental point-like particles. Their
centre-of-mass energy can be more accurately determined and collisions between
leptons produce a much cleaner environment, allowing for easier observation of
the resultant particles and simpler analysis of the data.
% since less interactions need to be taken into account.  TODO improve this
% explanation

% One of the drawbacks to circular accelerators, for lighter particles
% especially,
Circular accelerators, such as LEP, have a major drawback when accelerating
leptons, and that is the loss of a particle's energy due to synchrotron
radiation, which limits the focusing of electron beams~\cite{Oide1988ru}.  This
is the emittance of radiation from relativistic charged particles that are
moving in a uniform magnetic field. The energy loss is inversely proportional to
the fourth power of the rest mass of the particle \cite{sokolov1966synchrotron},
meaning that electrons lose more energy than protons by a factor of about
\num{e13} which is one of the main reasons why protons are being collided at
CERN rather than electrons. During experiments performed at the LEP the radiated
power when running at \SI{100}{\giga\electronvolt} reached about
\SI{18}{\mega\watt} which needs to be resupplied to the beam just to keep it's
circular trajectory.
% on-top of the power needed to accelerate the beam.
% TODO ^ TODO I'm not sure
% how much 18MW is.  TODO should I talk about synchrotron radiation in the
% theory section?

% TODO talk more about the goals and location?
This problem can be overcome by creating linear particle accelerators. There are
currently two radio-frequency (RF) linear lepton--lepton accelerator proposals,
the Compact Linear Collider (CLIC)~\cite{Linssen2012hp} and the International
Linear Collider (ILC)~\cite{Behnke2013xla} which are expected to reach collision
energies of up to several \si{\tera\electronvolt} and
\SI{500}{\giga\electronvolt} respectively.
% Both collaborations have recently joined efforts under the Linear Collider
% Collaboration~\cite{}.
With current RF technology, the continuous scaling of linear lepton--lepton
accelerators to higher and higher energies, requires greater accelerator lengths
extending to the \SI{100}{\kilo\meter} scale for collisions in the
\si{\tera\electronvolt} scale. Building accelerators of this scale is deemed
impractical for most situations due to geographical and financial limitations.
This urges the development of new technologies in order to continue pushing the
energy frontier of particle accelerators while scaling down their lengths.


